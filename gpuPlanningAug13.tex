\documentclass[12pt]{article} 
\usepackage{amssymb,amsfonts,amsmath}
\usepackage{color,graphicx}
\usepackage{siunitx}
%\sisetup{
%per-mode=fraction,
%fraction-function=\tfrac
%}
\usepackage{fixltx2e}
\usepackage[nomarkers,tablesfirst]{endfloat}
\usepackage[left=0.7in,right=0.7in,top=0.7in,bottom=0.7in]{geometry}
\newcommand{\eqn}[1]{(\ref{#1})}
\newcommand{\numberofpatientsLOOCV}{22 }
\linespread{1.6}
%http://latex2rtf.sourceforge.net/latex2rtf_1_9_19.html#Conditional-Parsing
%Starting with LaTeX2RTF 1.9.18, there is a handy method for
%controlling which content should be processed by LaTeX or by
%LaTeX2RTF . Control is achieved using the standard \if facility of
%TeX. If you include the following line in the preamble of your document 
%
%     \newif\iflatextortf
%Then you will create a new \iflatextortf command in LaTeX . TeX
%sets the value of this to false by default. Now, LaTeX2RTF
%internally sets \iflatextortf to be true, and to ensure that this
%is always the case, LaTeX2RTF ignores the command
%\latextortffalse. This means that you can control how different
%applications process your document by
%
%     \iflatextortf
%     This code is processed only by latex2rtf
%     \else
%     This code is processed only by latex
%     \fi
%Note that \iflatextortf will only work within a section; you
%cannot use this command to conditionally parse code that crosses
%section boundaries. Also, it will only work on complete table or
%figure environments. Due to the mechanism used by LaTeX2RTF in
%processing these environments, at this time the only way to
%conditionally parse tables and figures is to include two complete
%versions of the environment in question, nested within an
%appropriate \iflatextortf structure.
%
\newif\iflatextortf

\iflatextortf 
%do nothing
\else  %pdflatex
\usepackage{boxedminipage,float}
\usepackage{wrapfig,setspace}
\newcommand{\picdir}{pdffig}
\fi

\usepackage[pdftex, plainpages=false, colorlinks=true, citecolor=black, filecolor=black, linkcolor=black, urlcolor=black]{hyperref}


%=================================================
\begin{document}

\title{\bf \Large
A Model Evaluation Study for Treatment Planning of Laser Induced Thermal Therapy
}

\author{ S.~Fahrenholtz\textsuperscript{1,2},
         T.~Moon\textsuperscript{3},
         M.~Franco\textsuperscript{3},
         D.~Medina\textsuperscript{3}, 
         S.~Danish\textsuperscript{4}, \\
         A.~Gowda\textsuperscript{5},
         A.~Shetty\textsuperscript{5},
         F.~Maier\textsuperscript{1},
         J.~D.~Hazle\textsuperscript{1,2},
         R.~J.~Stafford\textsuperscript{1,2}, \\
         T.~Warburton\textsuperscript{3},
         D.~Fuentes\textsuperscript{1,2*}
       }
\date{ \small
\textsuperscript{1}The University of Texas M.D. Anderson Cancer Center,\\
Department of Imaging Physics, Houston, TX 77030, USA \\
\textsuperscript{2}The University of Texas at Houston, Graduate School of Biomedical Sciences,\\
Houston, TX 77030, USA\\
\textsuperscript{3}CAAM, Rice University, Houston, TX 77005, USA\\ 
\textsuperscript{4}Robert Wood Johnson, New Brunswick, NJ 08901, USA \\
\textsuperscript{5}BioTex, Inc., Houston, TX 77054, USA \\
\textsuperscript{*} Corresponding Author: \texttt{dtfuentes@mdanderson.org}   \\
%Webpage: \texttt{http://www.caam.rice.edu/~timwar/TimWarburton/Software.html}
}


\maketitle

%\doublespacing

\begin{abstract}
%Purpose
{\color{red}
A cross validation analysis evaluating computer model prediction accuracy
for \textit{a priori} planning magnetic resonance-guided laser induced
thermal therapy (MRgLITT) procedures in treating focal diseased brain
tissue is presented.
Two mathematical models are considered.
%Two mathematical models are selected to evaluate the
%trade-off between the time investment in the algorithm, efficiency of the
%numerical implementation, and accuracy required in predicting the final endpoint
%of the therapy.  
}
%Methods
(1) A spectral element discretization of the transient Pennes bioheat
transfer equation is implemented to 
%implemented on GPU computing architectures 
predict the laser induced heating in perfused tissue. 
(2) A 
%computationally inexpensive, 
closed-form algorithm for predicting the steady state heat transfer from a
linear superposition of analytic point source heating functions is also
considered.
{\color{red}
Prediction accuracy is
retrospectively evaluated 
via leave-one-out cross validation (LOOCV).  
Modeling predictions are quantitatively evaluated
in terms of a Dice similarity coefficient (DSC) between the
simulated thermal dose and
thermal dose information contained within
$N$ = \numberofpatientsLOOCV 
MR thermometry datasets.
%obtained during MRgLITT procedures. 
}
%Results
During LOOCV analysis, the transient model's DSC mean and median is 0.7323 and 0.8001, respectively,
with 15 of \numberofpatientsLOOCV DSC values exceeding the success criterion of DSC $\geq$ 0.7.
The steady state model's DSC mean and median is 0.6431 and 0.6770, respectively,
with 10 of \numberofpatientsLOOCV passing.
A one-sample, one-sided Wilcoxon signed rank test indicates that the
transient FEM model achieves the prediction success critera, 
DSC $\geq$ 0.7, at a statistically significant level. 
%The null hypothesis was accepted for the steady state model.
%Conclusions
%Further, the spectral element discretization of the transient bioheat transfer model
%provides greater potential in accurately calibrating the tissue properties needed
%in treatment prediction.
%under an Ockham razor philosophy preferential to simplicity,
%the steady state superposition model with fewer floating
%point operations would be selected with respect to outcome prediction.
%However, in the appropriate context, both approaches are likely to provide useful
%information in guiding treatment decisions.


\paragraph{Keywords} Bioheat Transfer $\cdot$ 
                     Laser Induced Thermal Therapy $\cdot$ 
                     MR Temperature Imaging $\cdot$ 
                     GPU Computing 
\end{abstract}

%%%%%%%%%%%%%%%%%%%%%%%%%%%%%%%%%%%%%%%%%%%%%%%%%%%%%%%%%%%%%%%%%%%%%%%%
%%%%%%%%%%%%%%%%%%%%%%%%%%%%%%%%%%%%%%%%%%%%%%%%%%%%%%%%%%%%%%%%%%%%%%%%
\section{Introduction}
%%%%%%%%%%%%%%%%%%%%%%%%%%%%%%%%%%%%%%%%%%%%%%%%%%%%%%%%%%%%%%%%%%%%%%%%
%%%%%%%%%%%%%%%%%%%%%%%%%%%%%%%%%%%%%%%%%%%%%%%%%%%%%%%%%%%%%%%%%%%%%%%%

Approximately 211,000 patients present each year in the US with brain
tumors.  Of these, 38,000 are benign primary tumors, 23,000 malignant
primary tumors, and 150,000 are metastatic, originating largely from lung,
breast, and melanoma~\cite{NCIFact13,
kalkanis2010evidence,gavrilovic2005brain,bafaloukos2004treatment,Owonikoko2014}.
The average life expectancy for patients with primary and metastatic
malignancies in the brain, from time of diagnosis until death, is
approximately 12-16 months. Five year survival is among the lowest of all
cancers. Current treatment options include conventional surgery,
stereotactic radiosurgery (SRS), or chemotherapy.  
Surgical resection may be preferred for patients presenting with a single,
solitary lesion, or lesions greater than 2.5 cm - 3.0 cm in which the size
or location of the tumor causes neurological symptoms such as seizures,
headaches, cognitive or motor deficits, that can be resolved by reducing
the volume of the disease.
SRS, such as Gamma Knife\textsuperscript{\textregistered}, is typically performed on patients with multiple
tumors, tumors under 2.5 cm in diameter, and deep seated tumors~\cite{alexander1995stereotactic}.
An additional target in brain is
epileptogenic foci for patients with medically refractory epilepsy,
where MRgLITT is being considered for the surgical armamentarium for those patients~\cite{curry2012mr,Nowell2014}.
Unfortunately, patients with malignant recurrences who have reached max radiation dose
limitations and complications with surgical resection create a group of
patients with no remaining conventional treatment options;
meanwhile patients with medically refractory epilepsy have limited interventions available.
Magnetic Resonance-guided Laser Induced Thermal Therapy (MRgLITT) presents
an alternative, minimally invasive thermal ablation technique for these
groups of patients and has been safely and successfully applied to 
each~\cite{carpentieretal08,carpentier2011laser,
carpentier2012mr,torres2013stereotactic,jethwa2012magnetic,
schwarzmaier2002basic,schulze2004laser}. 

Under MR guidance, the laser applicator is carefully navigated through
critical structures and placed directly into the diseased tissue to induce
ablative heating and destroy the tissue. 
Real-time thermometry of the treatment volume during
laser heating provides a mechanism by which it is possible to deliver
these therapies in both a safe and effective 
manner~\cite{rieke2008mr, denis2005magnetic,stafford2010magnetic,
woodrum2010feasibility} as well as estimate the extent of tissue
damage~\cite{hyperthermia2003basic,mcdannold2006uterine,mcnichols2004technical}. 
However, the heating induced by the laser is not constrained exclusively
to tumor tissue and nearby tissue damage is possible. 
For these procedures to progress to standard of care, 
\textit{a priori} determination of the optimal placement of the laser catheter(s) is crucial
for achieving a more conformal delivery of therapy over the target volume
with minimal comorbidity of intervening or adjacent tissue. 
%Currently, neuro-navigation software is used to guide placement. 

This manuscript focuses on the development of a practical methodology for
evaluating computer model predictions for \textit{a priori} planning the procedure  
given $N$ datasets from previous procedures.
%Evaluation focuses on the trade-off between (1) algorithmic and numerical
%implementation efficiency (2) against, most importantly, prediction accuracy.
{\color{red} Evaluation focuses on prediction accuracy.}
Retrospective analysis of MR thermometry data acquired during previous procedures is
essential to train or calibrate the computer model parameters. 
The machine learning and statistics community have a rich history in
applying various algorithmic and physics-based data models to reach conclusions from a
given dataset~\cite{Breiman2001,Hastie2005}. 
Here we assume that a Pennes bioheat transfer model~\cite{Pennes1948} provides
representative predictions of the  physical process underlying the heating
observed within the MR thermometry data.
This physics-based approach
provides a theoretically sound and concise methodology to statistically summarize the high
dimensional thermometry dataset with a low dimensional model parameter subset. 

%The computer modeling approach targets a practical clinical setting of the operating room 
%or surgical suite where portable computing hardware, efficient software implementations,
%and efficient algorithm implementations are needed to deliver responsive predictions
%of the treatment outcome.  
Two distinct modeling approaches are pursued: 
(1) A GPU implementation of an unstructured hexahedral spectral element method
for predicting the bioheat transfer is developed. 
(2) A computationally inexpensive algorithm for predicting the heat transfer
from a linear superposition of analytic point source heating functions is also
presented as a reference implementation.
{\color{red}
Combined with the data, the two modeling approaches presented provide an
environment to critically evaluate model accuracy and  selection for therapy
planning of MRgLITT. 
}
%``Essentially, all models are wrong, but some are useful'', Box~\cite{box1987empirical},
Both approaches 
build intuition in the prediction by
repetitively training the underlying physics model 
to statistically match representative datasets. 
Predictions are critically evaluated in terms of solution efficiency and
accuracy for prospective treatment planning of MRgLITT procedures. 
Leave-One-Out Cross Validation (LOOCV) is used to simulate the clinical
scenario in which $N$ datasets from previous procedures are available to
calibrate the computer model. LOOCV provides an objective framework to
critically estimate the accuracy and confidence in predicting the outcome
of the procedure for the $N+1$ patient~\cite{Stone1974,Geisser1975,Kohavi1995,Browne2000,Arlot2010}.

%===================================================================
\section{Methods }
\subsection{Thermometry Data }\label{ThermometryData}
%===================================================================

MR thermal monitoring from MRI guided stereotactic laser induced thermal
therapy (LITT) was considered in \textit{N} = \numberofpatientsLOOCV MR thermometry datasets.
The datasets were vetted for motion artifacts, low signal-to-noise, 
and catheter induced signal voids that would
spuriously reduce the modeling accuracy.
Each patient was treated with The Visualase Thermal Therapy System
(Visualase, Inc., Houston, TX).  The Visualase\textsuperscript{\textregistered}
system includes a 15 W
980 nm diode laser, a cooling pump, and a laser applicator set.
The laser applicator set is disposable and consists of 
a 400 \si{\micro\metre} core silica fiber optic with a cylindrical diffusing tip
housed within a 1.65 \si{\milli\metre} diameter saline-cooled polycarbonate cooling 
catheter~\cite{torres2013stereotactic,mcnichols2004technical}; see
Figure~\ref{LaserApplicatorOverview}.
Applicator cooling lines and laser fiber optics are connected through a
waveguide between the control room and the bore of the MR magnet. 
An MR compatible headholder is used to secure the patient's head.
The trajectory to the targeted tumor lesion was obtained 
using the Brainlab navigation system (Brainlab, Westchester, IL USA). 
A battery-powered hand drill was used to place
a threaded plastic bone anchor within the skull.
The laser applicator is secured to the threaded plastic bone anchor.
The Visualase\textsuperscript{\textregistered} system imports images from a 3D MPRAGE sequence
to verify applicator position within the lesion. 
The depth is determined by the navigation software and is input retrospectively
within this study. 

%~~~~~~~~~~~~~~~~~~~~~~~~~~~~~~~~~~~~~~~~~~~~~~~~~~~~~~~~~~~~~~~~
\begin{figure*}[p!]
\centering
\iflatextortf 
%do nothing
\else
\begin{tabular}{ccc}
\scalebox{0.4}{\includegraphics*{\picdir/OverviewApplicator}} \\ 
(a)\\
\scalebox{0.18}{\includegraphics*{\picdir/FEMesh}}  
&
\scalebox{0.24}{\includegraphics*{\picdir/powerProfile}}  
\\
(b) & (c)  \\
\end{tabular}
\fi
\caption{
(a) The Visualase\textsuperscript{\textregistered} applicator modeled in
this application and a diagram of the photon emitting diffusing
tip and the cooling fluid  are shown. (b) A finite element mesh conforms to the applicator
and is used as the template for the calculations. (c) A representative
time-temperature history profile of the thermometry data at two points within the brain tissue,
 $\sim$1 mm from the applicator, is shown. The corresponding power history is also shown. 
} \label{LaserApplicatorOverview}
\end{figure*}
%~~~~~~~~~~~~~~~~~~~~~~~~~~~~~~~~~~~~~~~~~~~~~~~~~~~~~~~~~~~~~~~~

MR temperature imaging was performed on a 1.5 T MRI
(GE Healthcare Technologies, Waukesha, WI)
using a 2D gradient echo sequence~\cite{staffordetal04} 
(FA = \SI{30}{\degree}, FOV = $24 \times 24$ cm\textsuperscript{2}, matrix size = $256 \times 128$, 
TR/TE=37.5/20 ms, receive-only head coil,  
5 sec. per update).
%No acceleration factor was used.
The imaging plane was chosen perpendicular to the axial direction of the applicator 
and allowed monitoring of critical structure regions.
The Visualase\textsuperscript{\textregistered} workstation communicates with the MR scanner to obtain
raw DICOM imaging data during the procedure.
The temperature dependent water proton resonance frequency shift is measured
by calculating the complex phase-difference observed during heating.
The water proton resonance frequency chemically shifts to lower
frequencies with higher temperatures (caused by rupture,
stretching, bending of hydrogen bonds)~\cite{ishihara1995paf}.
The total
temperature change, $\Delta u$, is proportional to the measured
phase change, $\delta \phi$.
\[
\Delta u = \frac{ \delta \phi } { 2 \pi \alpha \cdot \gamma B_0 \cdot \text{TE}}
\]
%$\alpha = -0.0097 \left[\frac{ppm}{^oC}\right]$ 
Here $\alpha$ is the temperature sensitivity coefficient, $\gamma$ is the
gyromagnetic ratio of water, $B_0$ is the static magnetic field strength, and
TE is the echo time.  
A baseline body temperature of \textit{u}\textsubscript{0} = \SI{37.0}{\degreeCelsius} is assumed to obtain
absolute temperature.
An Arrhenius rate process model~\cite{pearce1995rate}
was used to evaluate the thermal dose resulting from the
time-temperature history of the laser exposure.
\begin{equation}\label{ArrheniusDamage}
\text{$\Omega$}(t) = 
      \int_0^t \! Ae^{\frac{-E_A}{Ru(\tau)}} \, \mathrm{d} \tau
\end{equation}
In this Arrhenius thermal dose model, the frequency factor, $A$,
and the activation energy, $E_A$, are 
experimentally determined kinetic parameters. The values for $A$ and
$E_A$ were 3.1E98 {\color{red}s\textsuperscript{-1}} and 6.28E5 {\color{red}(J/mol)},
respectively, and have been used in previous
studies~\cite{carpentieretal08,schwarzmaier1998treatment,mcnichols2004mr}. 
$R$ is the universal gas constant.  The thermal dose was assumed to
be lethal at doses $\omega$  $\geq 1$ as seen in previous
reports~\cite{carpentieretal08,mcnichols2004mr}. 

%\begin{table}[h]
%\centering
%\caption{Patient Cohort Summary Statistics (N=\numberofpatientsLOOCV). MR 1.5 T scanner}\label{PatientSummaryStatistics}
%\begin{tabular}{c|ccc}
%                       &   Min   &  Mean & Max  \\\hline
%Age   (yr)             &   14    & 27    & 60   \\
%Power (W)              &   7     & 10    & 14   \\
%Damage Volume(mm$^3$)  &   1.1   & 2.2   & 2.0  \\
%\end{tabular}
%\end{table}

Prior to treatment delivery, a low power test pulse\textemdash\textit{e.g.}, 4 W for 30 s\textemdash
is applied to verify position of the diffusion fiber optic within the catheter.  
The test pulse is
sufficient to allow thermal visualization but not cause thermal damage. 
%Criteria for inclusion in our study included...
Multiple thermal imaging datasets are available per patient;
only the therapy delivery pulses are considered in this study.
A representative laser power profile used
during the therapy is shown in Figure~\ref{LaserApplicatorOverview}. 
All DICOM header information was imported into an SQLite database 
to provide efficient queries and 
organize thermometry data for reproducible analysis and processing. 
The schema provided by the Slicer~\cite{yeniaras2013design} DICOM
module was used as template for the table structure.
The object identifier (OID)
of the SQL table was used to provide anonymous references to the data.
The SQL functions,  \verb#group by#, \verb#group_concat#, and \verb#count#,
were used in quality assurance of the data
location, number of files, etc.
Metadata needed for the analysis includes:
\begin{enumerate}
\item The applicator orientation and heating region of interest (ROI)
are manually identified within the imaging datasets for input into the
computer models discussed in Sections~\ref{BioheatComputerModel} and \ref{AnalyticBioheatModel}.
\item A text file containing the relevant laser power history for each imaging
dataset is parsed and input into the simulation. The power history provides
information on the heating and cooling time intervals during the procedure.
\end{enumerate}



%===================================================================
\subsection{Simulation of Bioheat transfer within Laser Irradiated Tissue}\label{BioheatComputerModel}
%===================================================================

A time dependent Pennes~\cite{Pennes1948} bioheat transfer equation
provides a computer model for predicting the temperature field resulting
from the laser tissue interaction. 
\begin{equation} \label{BioheatPDE}
  \partial u_t   
 -\nabla \cdot  (   \alpha \nabla u  ) 
 + \frac{\omega c_\textit{blood}}{\rho c_p} (u  - u_a )
 = \frac{1}{\rho c_p} \; q_{laser}
     \qquad \text{in } U \backslash U_{tip}
\end{equation}
\[
   q_{laser}(x,t)  = 
    \int_{U_{tip}}
 \frac{p(t) \mu_\textit{eff}^2}{\text{Vol}(U_{tip}) }  
   \frac{ \exp(-\mu_\textit{eff} \| x -{ \xi}\|) }
      {4\pi \| x-{ \xi}\|} \; d\xi
   \quad   x \in U \backslash U_{tip}
\]
\[
  \alpha  = \frac{k}{\rho c_p}
\quad
  \mu_{eff}  = \sqrt{ 3 {\mu_a} \mu_{tr} }
\quad
 \mu_{tr} = \mu_a  + \mu_s (1-g)
\]
\[
n \cdot ( \alpha \nabla u  )  = 0 
                       \qquad \text{on } \partial U
\qquad
n \cdot ( \alpha \nabla u  )  = h (u - u_\textit{cooling}) 
                       \qquad \text{on } \partial U_\textit{tip}
\]
Here the tissue specific heat, $c_p$; tissue density, $\rho$;
thermal conductivity, $k$; perfusion, $\omega$; blood specific
heat, $c_{blood}$; 
and arterial blood temperature, $u_a$,
are deterministic model parameters 
obtained from literature~\cite{Handbook05,Welch95,duck1990}; 
see Table~\ref{modeldata}.
The laser source, $q_{laser}$, is a deterministic function of the
applied power, $p(t)$; optical
scattering, $\mu_s$; optical absorption, $\mu_a$; anisotropy factor, $g$;
and distance, $\|x-\xi\|$, from the source, $U_{tip}$.
Active cooling of the water flowing through the applicator is modeled 
by a Robin or mixed boundary condition in which the temperature flux at
the applicator interface is proportional to  the convection coefficient,
$h$, and the temperature difference between the cooling fluid
$u_\textit{cooling}$ and the tissue. A diagram of the 
Visualase\textsuperscript{\textregistered} applicator used
in this application is shown in Figure~\ref{LaserApplicatorOverview}.

%~~~~~~~~~~~~~~~~~~~~~~~~~~~~~~~~~~~~~~~~~~~~~~~~~~~~~~~~~~~~~~~~
\iflatextortf 
%do nothing
\else
\begin{table*}[p]
\caption{Constitutive data~\cite{Handbook05,Welch95,duck1990}}\label{modeldata}
\centering 

\begin{tabular}{|c|c|c|c|c|c|} \hline 
\textit{k} [$\frac{\si{\watt}}{\si{\metre\kelvin}}$]   & $\omega$ [$\frac{\si{\kilo\gram}}{\si{\metre\cubed\second}}$] &  \textit{g} [Unity]  &  $\mu_s$ [\si{\per\centi\metre}] &   $\mu_a$ [\si{\per\centi\metre}] & $\rho$ [$\frac{\si{\kilo\gram}}{\si{\metre\cubed}}$] \\ \hline
          0.527              &             9.0             & 0.88  &       14000              &       500             &  1045    \\ \hline
$c_{blood}$ [$\frac{\si{\joule}}{\si{\kilo\gram\kelvin}}$] &  $c_{p}$ [$\frac{\si{\joule}}{\si{\kilo\gram\kelvin}}$]& \textit{h} [$\frac{\si{\watt}}{\si{\kelvin\metre\cubed}}$] & $u_{cooling}$ [\si{\degreeCelsius}]& $u_a$ [\si{\degreeCelsius}] & \\ \hline
     3840                      &                  3600  & 100 & 21   & 37 & \\ \hline
\end{tabular}
%\begin{tabular}{|c|c|c|c|c|c|c|c|c|c|c|} \hline 
%\textit{k} [$\frac{\si{\watt}}{\si{\metre\kelvin}}$]   & $\omega$ [$\frac{\si{\kilo\gram}}{\si{\metre\cubed\second}}$] &  \textit{g} [Unity]  &  $\mu_s$ [\si{\per\centi\metre}] &   $\mu_a$ [\si{\per\centi\metre}] & $\rho$ [$\frac{\si{\kilo\gram}}{\si{\metre\cubed}}$] & $c_{blood}$ [$\frac{\si{\joule}}{\si{\kilo\gram\kelvin}}$] &  $c_{p}$ [$\frac{\si{\joule}}{\si{\kilo\gram\kelvin}}$]& \textit{h} [$\frac{\si{\watt}}{\si{\kelvin\metre\cubed}}$] & $u_{cooling}$ [\si{\degreeCelsius}]& $u_a$ [\si{\degreeCelsius}]\\ \hline
%          0.527              &             9.0             & 0.88  &       14000              &       500                  &  1045                        &            3840                      &                  3600  & 100 & 21   & 37    \\ \hline
%\end{tabular}
\end{table*}
\fi
%~~~~~~~~~~~~~~~~~~~~~~~~~~~~~~~~~~~~~~~~~~~~~~~~~~~~~~~~~~~~~~~~

An implicit Euler time discretization is used to reduce the 
time-dependent bioheat equation, Equation~\eqn{BioheatPDE}, into a sequence of elliptic problems. 
Hexahedral Lagrange elements (polynomial order = 3) were used in the finite
element discretization of the spatial domain.
These elements use a tensor-product of Gauss-Lobatto-Legendre (GLL) interpolation
nodes and are commonly referred to as spectral elements.  
A matrix-free preconditioned conjugate gradient algorithm is used to solve the
linear system of equations inherent to the discretization.  
An overlapping additive Schwarz preconditioner is used since the local
block problem on each element is well suited for the block-coupled
parallelism of the wide SIMD cores on the GPU.  
The matrix-free approach minimizes the storage
requirements and data movement of the finite element elliptic solvers.  The
preconditioned conjugate gradient algorithm does not explicitly demand the system
matrix to be stored but only requires the evaluation of matrix-vector products,
and the structure of the tensor-product hexahedral elements allow this action
to be computed with $\mathcal{O}$($N^4$) operations per degree-\textit{N} finite
element.  
%The actual performance depends on the degree of the finite
%element with higher degree elements.  
Avoiding assembly and storage of the stiffness matrix on the GPU
allows the solver to handle discretizations with a large number of elements to
compensate for the limited memory on the GPU.  This algorithmic approach is
shown to have high computational efficiency on the non-uniform memory
architecture of modern GPUs~\cite{Medina2014}.

All computations were performed on the template hexahedral mesh
shown in Figure~\ref{LaserApplicatorOverview} (b).
For each simulation, the template was registered to the observed
laser location for each patient.
The mesh consists of disjoint regions for the applicator and tissue.
A quadrilateral mesh was extruded axially along the applicator to create
the base of the hexahedral finite element mesh. The mesh for the tissue
conforms to the surface of the application and extends sufficiently far
to ensure that the boundary does not influence the heating.
The discretization consists of $N_{dof}=$844,032 total GLL nodes. 
The degrees of freedom across the volume of the applicator were removed;
the effect of the room-temperature cooling fluid which protects the laser fiber
during heating were considered through the boundary conditions at the surface
nodes. 
{\color{red}
A mesh resolution of approximately the 1mm pixel size near the applicator was used
to ensure convergence of the discretized solution~\cite{fuentesetal11a}. 
Multiple time resolutions were also evaluated to ensure convergence of the
time stepping scheme.
}

%%%%%%%%%%%%%%%%%%%%%%%%%%%%%%%%%%%%%%%%%%%%%%%%%%%%%%%%%%%%%%%%%%%%%%%%
\subsection{Analytic Steady State Solution}\label{AnalyticBioheatModel}
%%%%%%%%%%%%%%%%%%%%%%%%%%%%%%%%%%%%%%%%%%%%%%%%%%%%%%%%%%%%%%%%%%%%%%%%

A steady state version of the Pennes bioheat equation, Equation~\eqn{BioheatPDE},
was also considered as a surrogate model for the therapy planning, in order to
investigate the accuracy of a simpler, trained model.
Constant coefficients are assumed. A 1D spherically symmetric radial decomposition 
of the solution,
$u(r)~-~u_a~=~\frac{\mathcal{R}(r)}{r}$, simplifies the analysis of the differential operator in
spherical coordinates.
\[
    \frac{\omega c_\textit{blood}}{\rho c_p}  \frac{ \mathcal{R}(r)}{r} 
   -
    \frac{\alpha }{r}   \frac{d^2 \mathcal{R}}{dr^2}
   = 
    \frac{\mu_\textit{eff}^2 \;  P}{ \; \rho c_p}   
    \frac{ \exp \left(-\mu_\textit{eff} \; r \right) }{ 4 \; \pi \; r}
\]
\[
   u(r_1) = u_0
  \qquad
  \left. \frac{d u}{dr} \right|_{r_2}  = 0
  \qquad
   r_1 < r_2
\]
%FIXME { \color{red} May also consider Robin BC 
%\[
%  \left. k  \frac{d u}{dr} \right|_{r_1}  = h (u - u_\infty) 
%\]
%} 
From classical theory~\cite{boyce1992elementary}, 
the general solution is the linear combination of the homogeneous solution, $u_h$, and
a particular solution, 
$ u_p = \frac{1}{r} \left( A \exp (-\mu_\textit{eff} \; r) + B \; r \;  \exp
(-\mu_\textit{eff} \; r) \right) $.
In this case, the particular solution was obtained from the method of undetermined
coefficients for $A,B \in \mathbb{R}$. 
\begin{equation}\label{OneDSolution}
u = 
  \underbrace{
    C_1 \frac{\exp\left( \sqrt{\frac{\omega c_\textit{blood}}{k}} \; r \right)}{r}  
  + 
    C_2 \frac{\exp\left(-\sqrt{\frac{\omega c_\textit{blood}}{k}} \; r \right)}{r}  
    }_{u_h}
  + 
  \underbrace{
    \frac{\mu_\textit{eff}^2 \; P \; \exp \left(-\mu_\textit{eff} \; r \right) }
         { 4 \; \pi \; r \; \left(\omega c_\textit{blood}-k \; \mu_\textit{eff}^2 \right)} 
    }_{u_p}
  + 
    u_a
\end{equation}
The boundary conditions are used to determine the coefficients of the homogeneous solution.
Applicator cooling is specified by the boundary condition at $r = r_1$.
The domain is assumed large enough that no heat flux is observed at the far boundary $r = r_2$.
\[
\begin{bmatrix}
    \frac{\exp\left( \sqrt{\frac{\omega c_\textit{blood}}{k}} \; r_1 \right)}{r_1}  
  & 
    \frac{\exp\left(-\sqrt{\frac{\omega c_\textit{blood}}{k}} \; r_1 \right)}{r_1}  
  \\
  \left.
   \frac{d}{dr}
    \frac{\exp\left( \sqrt{\frac{\omega c_\textit{blood}}{k}} \; r \right)}{r}  
  \right|_{r_2}
  & 
  \left.
   \frac{d}{dr}
    \frac{\exp\left(-\sqrt{\frac{\omega c_\textit{blood}}{k}} \; r \right)}{r}  
  \right|_{r_2}
\end{bmatrix}
\begin{bmatrix}
  C_1 \\
  C_2 \\
\end{bmatrix}
= 
\begin{bmatrix}
      u_0 - u_p(r_1) - u_a    \\
  - \left. \frac{d u_p}{dr} 
  \right|_{r_2}\\
\end{bmatrix}
\]
%The 1D solution provides an estimate of the heating from a single point source
%with applicator boundary at $r_1$. The Symbolic Toolbox of MATLAB was used to determine
%and verify all coefficients and \verb#ccode# was used to write out the kernel.
%Similar analytical solutions are provided in \cite{Giordano2010,Vyas1992,Deng2002}.
The 1D solution provides an estimate of the heating from a single point source
with applicator boundary at $r_1$. Mathematica 7 (Wolfram, Champaign, IL) was used
to determine and verify all coefficients. Then, \verb#ccode# was used to write out the kernel.
Similar analytical solutions are provided in \cite{Giordano2010,Vyas1992,Deng2002}.
%FIXME \text{ \color{red} (Do we get the same result ?) } 
Heating caused by the cylindrical geometry of the diffusing tip was modeled as
evenly distributed point sources, $M = 10$, along the axial dimension of the
applicator at positions $r_{0_i}$. 
\begin{equation}\label{GreenSuperPosition}
  u = \sum_{i=1}^M u_h( r-r_{0_i}) + u_p( r-r_{0_i})
\end{equation}


%%%%%%%%%%%%%%%%%%%%%%%%%%%%%%%%%%%%%%%%%%%%%%%%%%%%%%%%%%%%%%%%%%%%%%%%
\subsection{Model Calibration}\label{ModelCalibration}
%%%%%%%%%%%%%%%%%%%%%%%%%%%%%%%%%%%%%%%%%%%%%%%%%%%%%%%%%%%%%%%%%%%%%%%%

For each thermometry dataset discussed in Section~\ref{ThermometryData},
an inverse problem was solved to calibrate both computer models considered 
to the observed heating.
Previous work~\cite{fahrenholtz2013generalised} 
showed that the optical parameters provide the highest sensitivity in the temperature
predictions over the range of physically meaningful model parameters.
Consequently, optical parameters are considered in the optimization.
The analytical form of the standard diffusion approximation for
the laser source term concisely represents the heating as a function
of the single optical parameter, $\mu_\textit{eff}$. {\color{red}At the end of
optimization for one MRTI dataset, the dataset has a corresponding
optimal $\mu_\textit{eff}$ that is constant in space and time.}


The out-of-plane translation component, $z$, of the mesh template 
shown in Figure~\ref{LaserApplicatorOverview} was also optimized.
The physics of the MR thermometry data acquisition averages
the temperature over the slice thickness and the translation update
is implemented to tune the registration of the computational domain to the MR thermometry data.
The remaining input parameters are assumed fixed.

% \[
%  f_1(\mu_\text{eff},z)
%    =  
%                  \sum_{k=1}^{N_\text{step}} \| u(t_k) - u^\text{MRTI}(t_k) \|^2
% \qquad
% \qquad
%  f_2(\mu_\text{eff},z)
%    = 1-DSC(\mu_\text{eff},z)
%  \qquad
%    DSC(\mu_\text{eff},z)
%    = 2 \frac{A \cap B}{A+B}
% \]

The DAKOTA (Sandia National Laboratories)~\cite{eldred2007dakota} library was
used to optimize $\mu_\textit{eff}$  for the transient and steady
state  models. 
The $L_2$ error over space and
time was used as the objective function.
\[
   (\mu_\textit{calib},z_\textit{calib})  = \text{arg}\min_{(\mu_\textit{eff},z)}  
                 \sum_{k=1}^{N_\textit{step}} \| u(t_k) - u^\textit{MRTI}(t_k) \|^2
\]
Thermometry data is denoted $u^\textit{MRTI}$.
All time steps were considered
for the transient analysis of the computer model presented
in Section~\ref{BioheatComputerModel}.
For the steady state model presented in Section~\ref{AnalyticBioheatModel},
the objective function was the $L_2$ norm between the model and the MRTI's maximum heating time point.
%The Dice similarity coefficient (DSC) was used as a measure of model prediction accuracy against
%damage observed in MR thermometry data. 
%For the transient model, regions of DSC overlap were considered as  
%Arrhenius dose, $\Omega$, $\geq$ 1.
%For the steady state model, DSC overlap was considered as regions where
%the MRTI max heating and model's temperature exceeded 57 $\degree$C.
%For some datasets, optimizing only the $L_2$ norm led to a DSC = 0, even if
%DSC $\geq$ 0 was possible. Therefore, a penalty term
%was added to the objective function that ensured the optimizer would
%converge to a $\mu_\text{eff}$ that had a DSC $\geq$ 0 if possible. 
A quasi-Newton optimization method, \verb#opp_q_newton#, was implemented as the
optimization algorithm for both models. 
% The method allows search directions of negative curvature.
% A BFGS preconditioner and  Eisenstat-walker convergence criteria was used for
% the inexact Newton solve.
Gradients of the objective functions were computed using numerical
finite differences. The calibration was solved as a bound constrained
optimization problem. A physically feasible parameter bound on the optimization of
the optical parameters, $\mu_\textit{eff} \in [0.8,400]$ m$^{-1}$, was obtained from
literature~\cite{Welch95}. During calibration, the initial value for $\mu_\textit{eff}$
was 180 m$^{-1}$. The initial value was calculated via the $\mu_\textit{eff}$ identity
from Equation~\eqn{BioheatPDE} and the $\mu_\textit{a}$, $\mu_\textit{s}$, and $g$ values
from Table~\ref{modeldata}.  The slice thickness of the MR thermometry data was used to bound the
optimization of the template out of plane translation.  


%%%%%%%%%%%%%%%%%%%%%%%%%%%%%%%%%%%%%%%%%%%%%%%%%%%%%%%%%%%%%%%%%%%%%%%%
\subsection{Leave-One-Out Cross Validation } \label{LOOCV_section}
%%%%%%%%%%%%%%%%%%%%%%%%%%%%%%%%%%%%%%%%%%%%%%%%%%%%%%%%%%%%%%%%%%%%%%%%

Leave-One-Out Cross Validation (LOOCV) is a method for estimating a
trained, \textit{i.e.} calibrated, model's accuracy in prediction ~\cite{Stone1974,Geisser1975,Kohavi1995,Browne2000,Arlot2010}.
Within this context, developing a `predictive model' refers to the process by which we
can confidently assign a probability to a treatment outcome, such as full tumor
destruction or damage to surrounding healthy tissue. 
Similar to the human cognitive process, the predictive computer model is built
from prior experience using MR temperature imaging data used to monitor the
procedure. The datasets are used to calibrate the computer model parameters as
discussed in Section~\ref{ModelCalibration}. The LOOCV algorithm is executed as: 
\begin{itemize}
\item for each thermometry dataset $i = 1, ..., N$
\begin{itemize}
\item  The average value for the optical coefficient, $\bar{\mu}_\textit{eff}^i$, 
is learned from the calibration results, $\mu_\textit{calib}^j$,
on the remaining $j \neq i$ datasets.
\[
   \bar{\mu}_\textit{eff}^i = \frac{1}{N-1} \sum_{j \neq i} \mu_\textit{calib}^j
\]
\item 
Tissue damage on the 
$i$-th dataset is predicted by using the average, $\bar{\mu}_\textit{eff}^i$, 
values from the $j \neq i$ cohort. 
The Dice similarity coefficient (DSC) provides an estimate of the agreement
with Arrhenius damage measured from thermometry data.
\[
   DSC(\bar{\mu}_\textit{eff}^i)   
    = 2 \frac{A \cap B}{A+B}
   \qquad \qquad 
   \bar{\mu}_\textit{eff}^i \neq \mu_\textit{calib}^j
\]
%The , $f_2$, was used to measure distance between the time-independent model,
%Section~\ref{AnalyticBioheatModel} and the 57$^o$C isotherm at the maximum
%heating seen in the MR thermometry. 
The DSC measures the area of overlap between
the area enclosed by the Arrhenius damage model for the thermometry data, $A$, and the
computer model prediction, $B$. 
The Arrhenius damage, Equation~\eqn{ArrheniusDamage}, is computed
from the simulated temperature field of the transient analysis. The isotherms
are used as the damage model for the steady state analysis.
Previous work in canine
brain demonstrated that the 57 \si{\degreeCelsius} isotherm produce damage regions similar
to the Arrhenius model for the ablation regime considered
in this study~\cite{yung2010quantitative}.
\end{itemize}
\end{itemize}

The trained model's predictive ability is
evaluated by analyzing the distribution of DSC values from the $N$ iterations of
LOOCV. 
A one-sample, one-sided Wilcoxon signed rank test examines if the trained
model prediction's median exceeds DSC $\geq$ 0.7. 
One-sample calculations were computed using a threshold DSC value, $0.7$, as 
the null hypothesis, $H_0$.
\[
   H_0: DSC  = 0.7  \qquad   H_1: DSC > 0.7
\]
{\color{red}
The value chosen is a commonly accepted value in image processing
literature~\cite{yung2010quantitative,Dice1945measures,zou2004three};  
$DSC=1$ implies complete agreement between the measure and predicted damage model,
$i.e.$ the predicted and measured damages volumes completely overlap.
}
A two-sample, paired Wilcoxon signed rank test was used to compare if the two models'
prediction medians are statistically different from one another.
All statistical tests and descriptive statistics were evaluated on
GraphPad Prism 6.01 (GraphPad Software, La Jolla, CA).

\begin{figure}[p!] 
\begin{tabular}{ccccc} 

\scalebox{0.16}{\includegraphics*{\picdir/dakota/0488/magnbestfit0059.jpg}} & \scalebox{0.257}{\includegraphics*{\picdir/model_vs_MRTI_SS/hot_0488_label.png}}  & \scalebox{0.32}{\includegraphics*{\picdir/model_vs_MRTI_SS/FEM_MRTI.png}} & \scalebox{0.295}{\includegraphics*{\picdir/model_vs_MRTI_SS/GF_MRTI.png}} & \scalebox{0.28}{\includegraphics*{\picdir/model_vs_MRTI_SS/dmg_legend.png}}
\\
(a) & (b) & (c) & (d) & (e) \\
\end{tabular}
\\
\caption{Representative thermometry data and calibrated model damage predictions.
(a) The magnitude of the complex valued thermometry data provides a visualization
of the anatomy and is provided as a reference. The applicator trajectory
is observed as a signal void in the image. The ROI displayed has a 
3.75 $\times$ 3.28 \si{\cm\squared} field of view and is shown in (b)-(d).
(b) MR thermometry at maximum heating is shown.
(c) FEM model predicted Arrhenius damage is compared to
Arrhenius damage based on MRTI.
(d) A comparison of the steady state damage model is shown.
The steady state damage model 
is the region enclosed by the \SI{57}{\degreeCelsius} isotherm. 
The color map indicates the geometrical overlap used in DSC calculations;
the legend is at right (e).
Respective DSC values for the FEM model (c) and steady state
model (d) are DSC = 0.8385 and DSC = 0.7442.
}  
\label{RepresentativeData}
\end{figure}



\begin{figure*}[p!]
\begin{tabular}{ccccc} 
\centering
\iflatextortf 
%do nothing
\else
%\begin{tabular}{cc}
%\scalebox{0.42}{\includegraphics*{\picdir/datasummarybestfit}}  
%&
\scalebox{0.55}{\includegraphics*{\picdir/hist_pics/FEM_Hist.png}} &  \scalebox{0.55}{\includegraphics*{\picdir/hist_pics/SteadyStateHist22.png}}  
\\
(a) & (b) \\
\fi
\end{tabular}
\caption{ 
Presented here are histograms of calibration analysis from the transient FEM model (a) and steady state model (b),
shown left and right; respectively. {\color{red}Both histograms have a bin width of \SI{13.0}{\per\metre}.}
The optical parameters, $\mu_\textit{eff}$,
recovered from each thermometry dataset considered is shown.
For each calibration, the bound constrained optimization was
restricted to a range obtained from literature, $\mu_\textit{eff} \in [0.8,400]$ m$^{-1}$. Leave-one-out cross validation was performed
using these \numberofpatientsLOOCV $\mu_\textit{eff}$ values.
The nominal value in brain tissue obtained from~\cite{Welch95} is 
$\mu_\textit{eff}$ = 180 m$^{-1}$.
} \label{QuantitativeComparison}
\end{figure*}


%~~~~~~~~~~~~~~~~~~~~~~~~~~~~~~~~~~~~~~~~~~~~~~~~~~~~~~~~~~~~~~~~
\iflatextortf 
%do nothing
\else
\begin{table*}[p]
\caption{Here are the descriptive statistics for $\mu_\textit{eff}$ during optimization
and DSC performance during optimization and LOOCV (\textit{N} = \numberofpatientsLOOCV).
The transient solve of the Pennes bioheat equation
using the Arrhenius damage, Equation~\eqn{ArrheniusDamage}, is denoted FEM.
Steady state analysis using the \SI{57}{\degreeCelsius} isotherm damage model is denoted GF.
Note that all DSC, skewness, and kurtosis quantities are unitless.
``\%-ile" refers to percentiles. \textit{E.g.}, 25\%-ile means the dataset's DSC performance exceeds
25\% of the population DSC values in ranked order.
}\label{DescriptiveStatistics}
\centering 
\begin{tabular}{c|cc|cc|cc} \hline 
Descriptive Statistic & FEM           & GF $\mu_\textit{eff}$ (m$^{-1}$) & FEM         & GF DSC opt. & FEM            & GF DSC LOOCV \\ \hline
      Minimum         &    67.26      & 138.4                            &   0.4865    &     0.3421   &      0        &    0.3312    \\ %\hline
      25\%-ile        &    115.5      & 167.2                            &   0.7356    &     0.5789   &      0.6709   &    0.5617    \\ %\hline
      Median          &    121.6      & 189.6                            &   0.8142    &     0.6925   &      0.8001   &    0.6770    \\ %\hline
      75\%-ile        &    146.1      & 212.4                            &   0.8652    &     0.7259   &      0.8429   &    0.7257    \\ %\hline
      Maximum         &    154.6      & 269.4                            &   0.8972    &     0.8476   &      0.8859   &    0.8143    \\ \hline
      Mean            &    125.6      & 194.7                            &   0.7824    &     0.6493   &      0.7323   &    0.6431    \\ %\hline
  Standard Deviation  &    21.78      & 38.42                            &   0.1091    &     0.1274   &      0.1930   &    0.1289    \\ \hline
      Skewness        &    -0.7906    & 0.5714                           &   -1.432    &     -1.190   &      -2.830   &    -1.150    \\ %\hline
      Kurtosis        &    1.026      & -0.2562                          &   1.609     &     1.142    &      9.919    &    0.6817    \\ %\hline
\end{tabular}
\end{table*}
\fi
%~~~~~~~~~~~~~~~~~~~~~~~~~~~~~~~~~~~~~~~~~~~~~~~~~~~~~~~~~~~~~~~~


%%%%%%%%%%%%%%%%%%%%%%%%%%%%%%%%%%%%%%%%%%%%%%%%%%%%%%%%%%%%%%%%%%%%%%%%
\section{Results}
%%%%%%%%%%%%%%%%%%%%%%%%%%%%%%%%%%%%%%%%%%%%%%%%%%%%%%%%%%%%%%%%%%%%%%%%

Representative thermometry images and calibrated computer model predictions are
shown in Figure~\ref{RepresentativeData}.  The measured and predicted thermal dose
is displayed. The Arrhenius damage model is shown for the transient model
predictions, Equation~\eqn{BioheatPDE}, and the thermometry data. The \SI{57}{degreeCelsius}
isotherm damage model is shown for the steady state analysis, Equation~\eqn{GreenSuperPosition}. 
Significant variability is seen in the heating due to local patient tissue heterogeneities,
tumor location, and nearby heat sinks in the brain such as large blood vessels
and cerebral spinal fluid (CSF). This is reflected by the non-ellipsoidal shape
of the isotherms and corresponding damage volume in the Arrhenius
estimates of the damage.  However, the calibrated thermal damage predictions show acceptable
agreement, \textit{i.e.} DCE $\geq$ .7~\cite{zou2004three}, between the measured and predicted
tissue damage in multiple patients. 

The calibration process applied to each thermometry dataset provides a
histogram of $\mu_\textit{eff}$ values in which the optimal agreement between
the model's prediction and the MR thermometry is observed for each model. 
The histogram of the $\mu_\textit{eff}$
values for both the transient and steady state model calibrations is shown in
Figure~\ref{QuantitativeComparison}. Literature values of the expected optical
properties is provided as a reference. 
Extrema of the feasible set are obtained
from the range of values observed in literature.
Descriptive statistics of the optimized $\mu_\textit{eff}$ values,
DSC during optimization, and DSC during LOOCV for 
both models is provided in Table~\ref{DescriptiveStatistics};
meanwhile, percentiles corresponding to interesting DSC thresholds
are presented in Table~\ref{DSC_and_percentiles}.


The overall DSC performance from both models during LOOCV analysis is
provided in Figure~\ref{ModelSelection}.
The performance is summarized by the
number of datasets that pass a given DSC threshold; the plot is
analogous to the Kaplan-Meier survival curve.
During LOOCV, the transient model had 15 of 22 datasets pass
the DSC $\geq$ 0.7 success criterion, while the steady state model
passed 10 of 22.


%~~~~~~~~~~~~~~~~~~~~~~~~~~~~~~~~~~~~~~~~~~~~~~~~~~~~~~~~~~~~~~~~
\iflatextortf 
%do nothing
\else
\begin{table*}[p]
\caption{Here are the percentiles that correspond to several interesting DSC thresholds.
The ``opt." and ``LOOCV" columns refer to the same groups of datasets described in Table~\ref{DescriptiveStatistics},
as well as the datasets plotted in Figure~\ref{ModelSelection}.
The smaller the percentile value, the better the model's performance.
`0' indicates all values pass at the given threshold; `100' indicates no values pass.
}\label{DSC_and_percentiles}
\centering 
\begin{tabular}{c|cc|cc} \hline 
DSC threshold (Unity) &  FEM (\%-ile)& GF DSC opt. (\%-ile)  &  FEM (\%-ile) & GF DSC LOOCV (\%-ile) \\ \hline
      0.4             &    0         &     8.357             &      5.776    &    7.176              \\ %\hline
      0.5             &    2.996     &     13.34             &      6.652    &    14.06              \\ %\hline
      0.6             &    11.36     &     27.34             &      12.97    &    27.52              \\ \hline
      0.7             &    15.25     &     57.64             &      30.35    &    58.61              \\ %\hline
      0.75            &    30.53     &     83.22             &      41.36    &    85.11              \\ %\hline
      0.8             &    44.36     &     95.12             &      49.97    &    96.14              \\ %\hline
      0.85            &    67.19     &     100               &      82.06    &    100                \\ %\hline
\end{tabular}
\end{table*}
\fi
%~~~~~~~~~~~~~~~~~~~~~~~~~~~~~~~~~~~~~~~~~~~~~~~~~~~~~~~~~~~~~~~~

\begin{figure*}[p]
\begin{tabular}{ccccc}
\centering
\iflatextortf 
%do nothing
\else
\scalebox{0.5}{\includegraphics*{\picdir/Survival_plots/opt_survival.png}} &  \scalebox{0.485}{\includegraphics*{\picdir/Survival_plots/LOOCV_survival.png}}  
\fi
\\
(a) & (b) \\
\end{tabular}
\caption{
Here, the overall predictive performance, measured by DSC,
is displayed for both models. (a) Left is the performance during
optimization, and the right is during LOOCV (b).
The horizontal axis displays increasing
DSC thresholds; the vertical axis displays the number of datasets
that pass the DSC threshold. Greater area under the curve (AUC)
indicates better prediction. In the FEM LOOCV plot, there
is one dataset that has a DSC = 0 and therefore does not appear
on the plot.
}
\label{ModelSelection}
\end{figure*}

The two-sample, paired Wilcoxon signed rank test rejected the null hypothesis
with a \textit{p}-value of 0.0059.
\textit{I.e.}, the difference of the transient and steady state models' medians of DSC
during LOOCV was statistically significant.
The one-sample, one-sided Wilcoxon signed rank test for the transient model
rejected the null hypothesis with a \textit{p}-value of 0.029.
\textit{I.e.}, the transient model's median of DSC during LOOCV was $\geq$
0.7 at statistically significant level.
The one-sample, one-sided Wilcoxon signed rank test for the steady state model
accepted the null hypothesis with a \textit{p}-value of 0.0732.
\textit{I.e.}, the steady state model's median of DSC during LOOCV was
not $\geq$ 0.7 at a statistically significant level.



%%%%%%%%%%%%%%%%%%%%%%%%%%%%%%%%%%%%%%%%%%%%%%%%%%%%%%%%%%%%%%%%%%%%%%%%
\section{Discussion}
%%%%%%%%%%%%%%%%%%%%%%%%%%%%%%%%%%%%%%%%%%%%%%%%%%%%%%%%%%%%%%%%%%%%%%%%

Given the number of available datasets, the LOOCV analysis
provides a methodology to recapitulate the clinical scenario
for treatment planning. Prior knowledge and experience is embodied
within the thermometry data from the previous $N$ ablations.
Information from the previous ablations is extracted by calibrating
a computer model to the available data. 
In this case, we calibrated our optical parameter to the thermometry data.
{\color{red}
%We are presenting this manuscript to introduce the possibility of a new
%paradigm of treatment planning via a train-and-predict scheme.  This method
%is becoming possible because of a growing number of retrospective patient
%datasets. 
%% Relatively simple models were chosen for this demonstration
%% to expedite the model training, \textit{i.e.}, optimization. 
Modeling goals are to ultimately utilize the calibrated model in optimizing
the thermal dose delivery.
The success
or failure of this paradigm is related to several factors.
First, there must be a sufficient quantity of retrospective datasets
for the training to converge. Second, the cohort of retrospective
datasets used in training must have sufficient similarity within the group
and to the prediction scenarios. Third, the model must be able 
to describe clinically relevant ablations. 
%In this investigation, none of these three factors are satisfied. 
%Nonetheless,
%we propose that model training may expedite clinically useful,
%patient-specific prediction by reducing the model complexity necessary
%and/or the need for patient-specific or population-based parameter
%measurements. 

}


If the LOOCV analysis only includes datasets that can be optimized to
have DSC $\geq$ 0.7, both models perform very well. This is because
the calibrated $\mu_\textit{eff}$ values have a much tighter distribution;
\textit{i.e.} datasets with an optimal DSC $\geq$ 0.7 had similar $\mu_\textit{eff}$ values.
Given this investigation is framed as a prediction on the $N$ + 1 patient,
cherry-picking the successful optimizations is inappropriate. Indeed, optimization results
with DSC $<$ 0.7 are included in the LOOCV analysis as seen in the descriptive statistics,
Table~\ref{DescriptiveStatistics}. However,
it is worth realizing that successful optimizations have similar
$\mu_\textit{eff}$ values and perhaps information beyond the
thermometry data would allow the calibration to grouped into similar cohorts.
\textit{E.g.}, additional meta information on the primary disease type, tumor location, and treatment history
would provide
useful information in the analysis that may demonstrate clustering during the calibration
and would further classify the tissue type.

The two models presented provides a canonical model selection comparison of the
trade-off between the time investment in the algorithm, efficiency of the
numerical implementation, and accuracy required for predicting the final endpoint
of the application. The two-sided, two-sample paired Wilcoxon signed rank test
indicates the predictive results' medians are significantly different and
only the transient model's median DSC performance was significantly
$\geq$ 0.7. 
{\color{red}
As a quantitative reference,
the forward solve mapping between model parameters, Table~\ref{modeldata}, 
and temperature field is considered the fundamental computational
operation of this study.
Total runtimes in the analysis are  proportional to the total number of foward solve
iterations in the optimation and LOOCV analysis. 
The run times for the forward solves in this study 
averaged 187.e3s and 11.2s for the transient model GPU implementation 
and steady state CPU implementation; respectively. 
These numbers are intended to provide intuition for the observed
practical run times on a local CPU 
% (Intel Xeon, 6 core, 2.4 GHz)  
(Intel Xeon, 6 core, 2.4 GHz, double precision peak 57 Gflops )  
workstation with attach GPU accelerators
%(NVIDIA Tesla M2070)
(NVIDIA Tesla M2070 double precision peak 515 Gflops) 
available to this study.
}
The finite element discretization of the governing equations requires significant
expertise of the GPU computing architecture as well as detailed algorithmic
understanding of both the finite element technology and matrix-vector multiply
within the iterative linear system solver to maximize floating point operation
throughput and minimizes memory transactions latencies.
As opposed to explicitly storing and reading matrix entries
from global memory, the matrix-free method recalculates the local matrix entries
needed within the linear system solve.
This memory access design pattern has demonstrated a 4-10x speedup  over
the matrix explicit methods~\cite{Muller2013,Knepley_2_2013,Medina2014}.
Meanwhile, the steady state superposition analysis, Equation~\eqn{GreenSuperPosition},
provides treatment predictions with fewer floating point operations and would be selected under an Occam
razor~\cite{jaynes2003probability} philosophy highly preferential to simplicity.
While all kernels
for the present spectral element methods and preconditioned conjugate gradient
method were hand coded in this manuscript, library implementations of the
matrix-free iterative solver approach are also
appearing~\cite{Knepley2013,Knepley2014}. 

%% {\color{red}
%% 
%% Previous real-time implementations of the kernel for
%% predicting the laser induced bioheat transfer were made feasible via
%% implementation on a massively parallel supercomputers~\cite{fuentesetal09,
%% Odenetal07,Dilleretal07,Fengetal07,Fuentesetal08}. 
%% Despite the increasing presence of large computational clusters within
%% large academic hospitals, reliance on such an approach makes clinical
%% translation of this research extremely difficult. This technology must
%% be practically available to any clinic using an MR-guided thermal therapy
%% system whereas integration with an institution's cluster and intranet creates
%% significant overhead and potential complications.
%% While modern single-chip multi-core CPU's have a peak performance of
%% $\mathcal{O}(10-100)$ GFLOPs and are sufficient 
%% for inexpensive algorithms such as the linear superposition of analytic functions presented
%% in Section~\ref{AnalyticBioheatModel}, 
%% current GPUs~\cite{Farber2011a,Muller2013} have a peak
%% performance of $\mathcal{O}(1-2)$ TFLOPs (single precision) and the
%% portability of these devices are ideal for developing more advanced 
%% numerical discretizations of the underlying partial differential equations
%% for guiding the therapy. However, a significant
%% time investment is needed to express the class of memory bound numerical
%% methods used in simulation within an algorithmic framework that can
%% simultaneously exploit these floating point capabilities and minimize the
%% global memory access latency of these devices.
%% % 28 = 1030/ (148/4) single precision.
%% % max floating point operations per second is 1030 GFlop
%% % data transfer of single precision numbers to the cores = bandwidth x data type size = 148  x (4bytes / float)
%% % >>> 1030./ (148./4.)
%% % 27.83783783783784
%% For example, the $\mathcal{O} (150)$~GB/s memory bandwidth of the Tesla M2070
%% used in the manuscript requires $\approx$ 30 floating point operations per
%% single precision global memory access to fully utilize the floating point capabilities
%% of the GPU. 
%% The spectral element method presented in Section~\ref{BioheatComputerModel}
%% achieves efficient global memory access through a matrix-free
%% implementation of the inherent linear system of equations.
%% 
%% }
%{\color{red}The focus of this manuscript is to compare two models' accuracy
%in predicting the maximum ablation size. However, for any given predictive model,
%a natural question to address is how much accuracy is necessary to clinically useful?
%A second question to answer is 
%The 

The finite element discretization of the governing equations, however, 
provides significant opportunity for further
physics-based improvements including 
higher order model spherical harmonic expansions in the laser fluence 
model~\cite{carp2004radiative}. 
As seen in Figure~\ref{QuantitativeComparison}, the models used
introduce a bias in the model parameter
recovery that differs from published literature values.
{\color{red}The bias may arise from inaccuracies in the modeling assumptions, perhaps
most significant being the use of optical tissue properties that are
invariant in space and time. The literature has clear examples
where temperature/damage dependent and spatially/temporally dependent
parameters are critical to the
prediction~\cite{Mohammed05,Schutt2008}. The use of thermometry data during
only the test pulse is also expected to influence the recovered optical
properties.
The choice to use a single constant parameter for full time history of each
MRTI dataset is motivated by
the practical model training focus of this investigation.
Higher order physics
models of the fluence are be expected to provide the highest accuracy in
recovering the optical parameters during the calibration process and for
characterizing the tissue properties.


%Higher order methods in which constitutive nonlinearites and different
%subsets of training sets such as the patient specifi
%test pulse that are optimal for planning
%is considered future work. This requires overfitting studies, harmonization
%of meta data and studies on how much training data is needed to achieve
%the desired accuracy.
 }

The finite element discretization of the governing equations
also provides a rigorous physics-based methodology to 
incorporate tissue heterogeneities into the treatment prediction.
Calibrations of the spatially heterogeneous 
optical parameter field has been shown to provide highly accurate 
predictions~\cite{fuentes2013inverse}.
{\color{red} The inclusion of spatially varied, damage dependent, and
multiple parameters should be pursued in future efforts.}
Adjoint-based methods of computing the gradient of the objective 
function are necessary to efficiently optimize in the higher dimensional parameter space.
Tissue heterogeneities could similarly be incorporated
into the steady state superposition analysis, Equation~\eqn{GreenSuperPosition}, in a patient-specific
`\textit{ad hoc}' manner. However, this would violate the underlying homogeneous
tissue parameter assumptions which provided the mathematical
structure for the concise analytical solution.
A Gaussian process~\cite{rasmussen2006gaussian,Constantinescu2013} 
framework may be appropriate
in which the model parameters recovered may be interpreted as hyperparameters
for the covariance kernels.
Similar to physics-based model calibration, optimization of the
hyperparameters in the Gaussian process kernels offers a complementary
trade-off between data fitting and smoothing. Further, Gaussian processes allow
for prior information to be used and provide a full probabilistic prediction
and an estimate of the uncertainty. 

Model calibration and training was limited to thermometry data in these efforts. 
The predictive capabilities are expected to improve with more information
provided by pre-treatment MR imaging such as dynamic contrast enhanced imaging
or perfusion imaging to help guide the selection of the model parameters,
especially if these preoperative images can inform the optical parameters.
Incorporating tissue heterogeneities into the model predictions would also
require segmentations of the neuro-anatomy as a template for the regional heterogeneity.
Each direction would benefit from the forethought of including these data
acquisitions into the therapy protocol. 
For example, current brain segmentation techniques~\cite{Menze2014} require high resolution
FLAIR, T2, and T1 imaging with and without contrast.
The SQL database used in organizing all data was vital to the reproducibility
of the analysis in these efforts; this additional information must be
incorporated.
Tools for communicating with the neuro-navigation software to locate the fiber
would also provide further information to improve the analysis throughput
and reproducibility.
Passive tracking of the applicator location using fiducials placed on the fiber
would additionally provide the registration information needed to align the
computational domain of the mathematical model. 


\paragraph{Conclusion}
Currently, the neurosurgeon reviews anatomical MR images to plan his/her
trajectory to reach the tumor with neuro-navigation software, but does not
have the capability to visualize outcomes of the laser ablation using
various trajectories beforehand. 
Fully developed and commercially implemented predictive computer models 
will extend this functionality to include \textit{a priori} visualization 
{\color{red} and optimization} of the
potential outcomes for complex treatment scenarios (multiple
applicators/trajectories) in which the laser ablation is performed near a
critical structure within a heterogeneous tissue environment. 
This work presents a step in this direction and demonstrates
the feasibility in establishing a confidence in these predictions.
Consideration of other available metadata to improve prediction accuracy
is a topic of on-going research.


%%%%%%%%%%%%%%%%%%%%%%%%%%%%%%%%%%%%%%%%%%%%%%%%%%%%%%%%%%%%%%%%%
\section*{Acknowledgments}

The research in this paper was supported in part through the O'Donnell Foundation,
NSF AIR-1312048, NIH 1R21EB010196-01, TL1TR000369, DoD W81XWH-14-1-0024,
 and CA79282 as well as the Cancer Center Support
Grant CA016672.
Research was jointly conducted at the MD Anderson Center for Advanced
Biomedical Imaging in-part with equipment support from General Electric
Healthcare.
The authors would like to thank Roger McNichols, PhD 
for his assistance in clinical laser ablation as well as the
DAKOTA~\cite{eldred2007dakota},
ITK~\cite{ITKSoftwareGuideSecondEdition}, 
Paraview~\cite{Paraview}, and
CUBIT~\cite{cubit} communities for providing enabling software for
scientific computation and visualization.  
%%%%%%%%%%%%%%%%%%%%%%%%%%%%%%%%%%%%%%%%%%%%%%%%%%%%%%%%%%%%%%%%%



%%%%%%%%%%%%%%%%%%%%%%%%%%%%%%%%%%%%%%%%%%%%%%%%%%%%%%%%%%%%%%%%%
\section*{Declaration of Interest}

At the time of this work, Visualase, Inc. and BioTex, Inc. provided
the surgical device technology and services. Drs. Shetty and Gowda
were employed by those companies. Since that time, Medtronic, Inc.
has acquired Visualase, Inc., and employs Dr. Shetty. BioTex, Inc.
employs Dr. Gowda. Dr. Danish has received educational honoraria
from Medtronic, Inc.

%%%%%%%%%%%%%%%%%%%%%%%%%%%%%%%%%%%%%%%%%%%%%%%%%%%%%%%%%%%%%%%%%


% BibTeX users please use one of
%\bibliographystyle{vancouver}
%\bibliographystyle{plain}  % Here the bibliography                 
\bibliographystyle{unsrt}  % chronological order
\clearpage
\bibliography{gpuPlanningAug13}

%\appendix
%\section{Cooling Model}
%The quantitative value of the convective cooling coefficient at the
%applicator interface $U_\text{tip}$ was derived using 
%non-dimensionalization arguments \cite{fasano2010mathematical}.
%\marginpar{\color{red}\tiny Moon, how do we derive the convective 

%coefficient $h$?}

\end{document} 
%%%%%%%%%%%%%%%%%%%%%%%%%%%%%%%%%%%%%%%%%%%%%%%%%%%%%%%%%%%%%%%%%%%%%%%%%%%%%%%%%%%%%%%%%%%%
%%%%%%%%%%%%%%%%%%%%%%%%%%%%%%%%%%%%%%%%%%%%%%%%%%%%%%%%%%%%%%%%%%%%%%%%%%%%%%%%%%%%%%%%%%%%
%%%%%%%%%%%%%%%%%%%%%%%%%%%%%%%%%%%%%%%%%%%%%%%%%%%%%%%%%%%%%%%%%%%%%%%%%%%%%%%%%%%%%%%%%%%%
%%%%%%%%%%%%%%%%%%%%%%%%%%%%%%%%%%%%%%%%%%%%%%%%%%%%%%%%%%%%%%%%%%%%%%%%%%%%%%%%%%%%%%%%%%%%
%%%%%%%%%%%%%%%%%%%%%%%%%%%%%%%%%%%%%%%%%%%%%%%%%%%%%%%%%%%%%%%%%%%%%%%%%%%%%%%%%%%%%%%%%%%%
%%%%%%%%%%%%%%%%%%%%%%%%%%%%%%%%%%%%%%%%%%%%%%%%%%%%%%%%%%%%%%%%%%%%%%%%%%%%%%%%%%%%%%%%%%%%
%%%%%%%%%%%%%%%%%%%%%%%%%%%%%%%%%%%%%%%%%%%%%%%%%%%%%%%%%%%%%%%%%%%%%%%%%%%%%%%%%%%%%%%%%%%%
%%%%%%%%%%%%%%%%%%%%%%%%%%%%%%%%%%%%%%%%%%%%%%%%%%%%%%%%%%%%%%%%%%%%%%%%%%%%%%%%%%%%%%%%%%%%
%%%%%%%%%%%%%%%%%%%%%%%%%%%%%%%%%%%%%%%%%%%%%%%%%%%%%%%%%%%%%%%%%%%%%%%%%%%%%%%%%%%%%%%%%%%%
%%%%%%%%%%%%%%%%%%%%%%%%%%%%%%%%%%%%%%%%%%%%%%%%%%%%%%%%%%%%%%%%%%%%%%%%%%%%%%%%%%%%%%%%%%%%
%%%%%%%%%%%%%%%%%%%%%%%%%%%%%%%%%%%%%%%%%%%%%%%%%%%%%%%%%%%%%%%%%%%%%%%%%%%%%%%%%%%%%%%%%%%%
%%%%%%%%%%%%%%%%%%%%%%%%%%%%%%%%%%%%%%%%%%%%%%%%%%%%%%%%%%%%%%%%%%%%%%%%%%%%%%%%%%%%%%%%%%%%


Big data analysis tools were needed.  Visualization of model comparisons to the
experimental data is automated observe potential inaccuracies. 

Continual iterative refinement of the input simulation setup parameters such as
the orientation, power history, and ROI are ongoing due to manual input errors.

The user interface allows the monitoring  damage
threshold temperatures within specific target anatomy. 
Triggered threshold temperature automatically deactivates the
laser and provides a safety mechanism for overly aggressive 
treament plans.

The data selection process was carefully chosen as to not bias the results.
However, in the case where a relatively simple model is a poor emulation of the
complex physics the model unjustifiably biases the conclusions.
In this case uncontrolled heating from tissue heterogeniety, out of plane
positioning, susceptibility, etc did not fit the modeling assumptions
and are excluded.

future work heterogeneous tissue properties and calibration.
imaging biomarkers of the important tissue heterogenieties.

The LOOCV methodology may also be extend to training and predicting
the thermometry data within a real-time data assimilation paradigm~\cite{fuentesetal12a}.

10 of the 100 top computers are using MIC~\cite{intelmic} processing chips~\cite{Medina2014}.
Further work will also include extending presented methodologies
to utilize the MIC for computing the therapy outcome.

Further, upon applicator placement,
updates to the treatment prediction (<30sec per prediction) are needed to
critically re-assess actual therapeutic outcomes possible against the
planned procedure to guide final delivery. 

A planning tool will enable the neurosurgeon to visualize various
trajectories with respect to patient-specific imaging and simulate complex
treatment scenarios. This would strengthen the confidence in aggressive
treatment plans and provide the ability to (1) get maximum coverage of the
tumor; (2) avoid heat near motor and visual critical structures – nerves,
vessels or functional areas; and (3) evaluate spread of heat
pre-placement, near heat sinks like the ventricles and large blood
vessels. The Visualase procedure entails a neurosurgeon stereotactically
inserting a pencil-lead thin laser catheter into the brain tumor, and then
heating the targeted tissue with laser energy until the tumor is dead.

Surgical technique All LITT procedures were performed using the intraoperative
3.0T MR scanner (IMRIS, Manitoba, Canada).  Acquisition of a fiducial-based
volumetric MRI was performed on all patients and fiducialbased A drill hole
through the skull and dura was created using a 3.2-mm drill bit on a
batterypowered hand drill.  A threaded plastic bone anchor was then screwed
into the twist-drill hole in the trajectory of the lesion. 

At the end of treatment, contrastenhanced T1-weighted VIBE sequence was
acquired for ablation volume verification. Upon completion, scalp closure
typically required only a single suture layer.  

A second or third overlapping thermal ablation zone
was sometimes created by withdrawing the fiberoptic applicator inside the
cooling catheter and repeating the ablation. 

Success is critically dependent on the
ability to destroy tumor tissue only, sparing important critical
structures.  If the actual
treatment-time placement differs from the planned trajectory, the proposed
planning software would allow the surgeon to reassess the ablation, with
respect to the new location, and decide if it still covers the target or
needs repositioning. This knowledge further enables the clinician to
refine expectations about achievable ablation and avoids accidental
over-treatment. This knowledge would also allow the surgeon to determine
if the ablation will cover the targeted lesion without affecting critical
structures and if tails of the tumor will be left untreated with the
current applicator placement. 

Computational tools are needed for planning the procedure to assist in
delivering maximum coverage of the tumor, avoid heat near motor and visual
critical structures - nerves, vessels or functional areas, and evaluate
spread of heat pre-placement, near heat sinks like the ventricles and
large blood vessels~\cite{fuentesetal11a,fahrenholtz2013generalised}. 


Predictive prospective computer simulation
integrated with 3D visualization and assessment of diagnostic and
interventional MR imaging information can provide useful information on
probe placement (number and location) in order to best assure the ability
to complete surgical laser ablation procedures.

Embedding these results within
uncertainty quantification methods may guarantee the
real-time predictability of computer simulations of the bioheat transfer
\cite{fahrenholtz2012f,fahrenholtz2013generalised}. 

Inaccuracies are currently arising from modeling assumptions that do not
match
physical reality. For the data shown in
Figure~\ref{RepresentativeData}.
modeling assumes homogeneous parameters. We are currently exploring
reproducible
image processing methods to identify heterogeneous tissue parameters that
are thermally
and optically distinct. This additional information is expected to improve
prediction accuracy.
Functionality to handle tissue parameters currently exists within
brainNek.

In addition to planning, computational methods are being investigated at
several time scales of MR Guided Laser Induced Thermal Therapy (MRgLITT)
procedures for assisting in additionally monitoring and controlling the
therapy outcome.
Computational techniques have been recognized as a potentially useful tool
to improve the quality of the thermal therapy planning for several
delivery modalities, including
laser,~\cite{chen2009optimizing,Fuentes2010,schwarzmaier2002basic,DeGreef2011,MicrowavePlanning,Prakash2012}.
Moreover, incorporating
real-time model based predictions into the temperature imaging
used for monitoring~\cite{potocki1993concurrent,todd2010model,
fuentesetal12a,roujol2011robust} 
provide a means to robustly estimate 
lost information due to data corrupting  
motion, low signal-to-noise-ratio (SNR), excessive heating, and 
catheter induced signal voids.
These enhanced real-time monitoring approaches 
may further assist in more safely and accurately 
controlling therapy delivery for increased efficacy 
of the procedure~\cite{mougenot2009three,Fuentesetal08}.  
Thermal therapy can also be applied in a manner complimentary with
conventional approaches and can be used to treat additional focal lesions
outside the range of conventional therapy, recurrence or be used
synergistically with local drug delivery
\cite{schwarzmaier2006mr,paiva2005intratumor}. 

A matrix-free preconditioned conjugate gradient solver was implemented to
minimize the storage requirements and data movement of the finite element
solver. Avoiding assembly and storage of the stiffness matrix on GPU
computing architectures allows the solver to handle discretizations with a
large number of elements and compensate for the limited memory on the GPU.

Potential phase change measurement error due to temperature and spatially
dependent temperature gradient induced changes in magnetic
susceptibility~\cite{peters2000heat,salomir2003fast} are not considered in this
measurement model.  However, as presented in the discussion, this approximation
is expected to have an minimal impact on the results of this work due to the
error metric used, Equation~\eqn{weightedL2Norm}.  


gNek Implementation of the bioheat transfer solver: The Nek simulation
tool was developed by Paul Fischer at Argonne National Laboratory to solve
the thermo-fluidic incompressible Navier-Stokes equations in
three-dimensional domains and has demonstrated excellent scalability on
clusters with conventional CPU compute cores of O($10^6$).  

Relevant to this project, gNek solves the screened Coulomb potential problem
discretized with high-order hexahedral finite elements.  

The coarse grid solver is handled on the CPUs as they are otherwise not
occupied by computation.  The initial implementation of gNek achieves up to 400
GFLOPS on NVIDIA GTX 590 GPU.  

Kovasznay flow [33] and the time-dependent solution by Ethier et al [34] were
used to test steady-state and time-dependent solutions, both being common
test-cases when simulating Navier-Stokes equations

Registration|(0.1275, 0.119, 0, 90, -4, 0)|Quality|1|VOI|[120, 150, 110, 145, 0, 0]|Power|[['Use'], ['Default'], 5.0, [19, 60], [0, 18]]|SNRROI|[60, 70, 60, 70, 0, 0]|NumberImagesCount|300|NumberSliceLoc|2|ManualStudyInfo|Unknown|Series Description|THERAML SAG OBLIQUE T13D SPGR||Manufacturer|GE MEDICAL SYSTEMS|Image Type|['ORIGINAL', 'PRIMARY', 'OTHER']|LogFilePower| if( time< 110.000000  ) return 0.000000;      else if( time< 126.000000  ) return 0.000000;   else if( time< 151.000000  ) return 3.000000;   else if( time< 175.000000  ) return 0.000000;   else if( time< 199.000000  ) return 3.000000;   else if( time< 320.000000  ) return 0.000000;   else if( time< 329.000000  ) return 0.000000;   else if( time< 355.000000  ) return 10.050000;   else return 0.0;|Echo Number(s)|1|Study Date|20100907|Echo Time|10|Repetition Time|22.5|Magnetic Field Strength|1.5|Pixel Bandwidth|78.125|Spacing Between Slices|3|Slice Thickness|3|Number of Averages|1|Flip Angle|30|Percent Phase Field of View|100|Acquisition Matrix|[0, 256, 128, 0]|Columns|256
